\documentclass[a4paper, 14pt]{extarticle}
\usepackage[left=2cm,right=2cm,top=2cm,bottom=2cm,bindingoffset=0cm]{geometry}
\usepackage[utf8]{inputenc}
\usepackage[english, russian]{babel}
\usepackage{amssymb, latexsym, amsmath}
\usepackage{indentfirst}
\usepackage{graphicx}
\usepackage{citehack}
\usepackage{tabularx}
\usepackage{listings}
\usepackage{pdfpages}

\lstloadlanguages{HTML,python}
\lstset{extendedchars=false,
  breaklines=true,
  breakatwhitespace=true,
  keepspaces = true,
  tabsize=2
}


\begin{document}
\begin{titlepage}

\newpage

\begin{center}
Московский Авиационный Институт \\*
(национальный исследовательский университет) \\*

\vspace{2em}

Факультет прикладной математики и физики \\*
Кафедра вычислительной математики и программирования

\vspace{10em}

\Large \textbf{Курсовой проект \\*
по дисциплине <<Информационные технологии в проектировании и производстве>>} \\*

\vspace{3em}

<<Разработка проектного офиса <<Цифромед>>
\end{center}

\vspace{8em}

\hspace{25em}\vbox{
  \hbox{\bfseries{Выполнил:}}
  \hbox{\hspace{1em} Данилычев И.\,А.}
}

\vspace{2em}

\hspace{25em}\vbox{
  \hbox{\bfseries{Руководитель:}}
  \hbox{\hspace{1em} Скородумов С.\,В.}
}

\vspace{\fill}

\begin{center}
Москва, 2015
\end{center}

\end{titlepage}

\newpage


\tableofcontents
\newpage


\section{Введение}
Планирование -- один из ключевых видов деятельности, необходимых для успешного менеджмента и реализации любого проекта, когда речь заходит о бизнесе. Целью данной работы являлось создание веб-сервиса, оформленного в виде SaaS-решения и представляющего собой проектный офис для совместной работы проектных команд.

В работе представлен процесс проектирования архитектуры и анализа бизнес-логики реализуемого приложения; кроме того, приведено краткое описание проектирования интерфейса (клиентской части). Всё это входило в мою часть работы, определённую через Work Breakdown Structure при планировании проекта.


\begin{figure}[!htb]
  \centering
    \includegraphics[scale=0.25]{../shared_images/wbs.png}
   \caption{WBS}
    \label{fig:start}
\end{figure}

\newpage

\section{Проектирование приложения}
\subsection{Схема БД}

\begin{figure}[!htb]
  \centering
    \includegraphics[scale=0.6]{../shared_images/schema.png}
   \caption{Схема БД}
    \label{fig:start}
\end{figure}

Как видно из схемы, ключевой сущностью является проект, который может иметь в своём составе одну или более задач. Задача, в свою очередь, связана с конкретным пользователем, ответственным за её сдачу, с помощью поля {\tt assigned\_to}.

У пользователей имеется возможность оставлять комментарии к задачам, используя реализованную на сайте систему комментариев. Каждый пользователь может как оставлять комментарии, так и не иметь их вообще.

Несмотря на произошедший позднее отказ от реляционных баз данных и переход к БД, основанных на технологии NoSQL, когда вся база представлена коллекцией документов различных типов без разбиения их на таблицы, спроектированная схема легла в основу финальной версии приложения.

\subsection{Анализ места размещения бизнес-логики}
\begin{figure}[!htb]
  \centering
    \includegraphics[scale=0.6]{../shared_images/business-logic/client.jpg}
   \caption{Клиентское приложение}
    \label{fig:start}
\end{figure}

На настольных приложениях бизнес-логика содержится на одном звене со всеми остальными слоями. Поскольку нет необходимости разделять слои, они зачастую перемешаны и не имеют четких границ.

\begin{figure}[!htb]
  \centering
    \includegraphics[scale=0.6]{../shared_images/business-logic/client-server.jpg}
   \caption{Приложение с архитектурой «клиент-сервер»}
    \label{fig:start}
\end{figure}


В клиент-серверном приложении имеются два звена, что приводит к созданию как минимум двух слоев. На начальном этапе сервер рассматривается только как удаленная база данных, и деление совпадает с рисунком -- приложение на клиенте и данные на сервере. Обычно вся бизнес-логика находится на клиенте, перемешанная с остальными слоями, такими как пользовательский интерфейс.

Достаточно быстро стало понятно, что можно сократить нагрузку на сеть и централизовать логику для уменьшения постоянных затрат на развертывание, перенеся большую часть бизнес-логики на сервер. Архитектурно сервер является хорошо подготовленным местом в клиент-серверной системе, хотя база данных как платформа даёт мало возможностей. БД были спроектированы для хранения и выдачи и в их архитектуру не были заложены возможности расширения в направлении бизнес-логики. Языки хранимых процедур в базах данных были разработаны для базовых преобразований данных, чтобы поддержать то, на что не хватало SQL. Языки хранимых процедур разработали для быстрого исполнения, а не для обслуживания сложных задач бизнес-логики.

Тем не менее, во избежание переусложнённости клиента часть бизнес-логики была перемещена в хранимые процедуры.

\begin{figure}[!htb]
  \centering
    \includegraphics[scale=0.6]{../shared_images/business-logic/client-server-2.jpg}
   \caption{Часть логики хранится на сервере}
    \label{fig:start}
\end{figure}


Когда проблема клиент-серверной архитектуры стала явной, возросла популярность 3-х звенного подхода. Наибольшей и самой тяжелой проблемой того времени было количество подключений. Сейчас многие базы данных могут обрабатывать тысячи единовременных подключений, однако десятилетие или два назад большинство баз данных падало где-то на пятиста подключениях.

Стало популярным объединение подключений в пул, однако для реализации пула подключений в системе с множеством отдельных клиентов, необходимо внедрить третье звено между клиентом и сервером. Среднее звено так и стало называться «среднее звено». В большинстве случаев среднее звено существовало только для управления пулом соединений, но в некоторых случаях бизнес-логика начала перемещаться в среднее звено потому, что языки разработки (C++, VB, Delphi, Java) гораздо лучше подходили для реализации бизнес-логики, чем языки хранимых процедур. Вскоре стало очевидно, что среднее звено -- это наилучшее место для бизнес-логики, и схема приложения трансформировалась:

\begin{figure}[!htb]
  \centering
    \includegraphics[scale=0.6]{../shared_images/business-logic/client-server-business.jpg}
   \caption{Пустой слой бизнес-логики}
    \label{fig:start}
\end{figure}

В таких случаях бизнес-слой не содержит бизнес правил. Это не настоящий бизнес-слой, а только форматтер XML (или другого потокового формата) и адаптер наборов данных базы данных. Хотя некоторые плюсы, такие как пул соединений и изоляция БД, могут быть достигнуты, это не настоящий слой бизнес-логики, скорее, инородный физический слой без слоя логики.

\begin{figure}[!htb]
  \centering
    \includegraphics[scale=0.6]{../shared_images/business-logic/client-server-business-2.jpg}
   \caption{Слой бизнес-логики, частично разгружающий БД}
    \label{fig:start}
\end{figure}

Обычно некоторые бизнес-правила приложения переходят в бизнес-слой, но то, что было в базе данных, так в ней большей частью и остается. При повторном использовании бизнес-слоя в таких разработках бизнес-правила должны повторяться и в клиентском приложении. Это сводит на нет основную цель внедрения бизнес-слоя.

\begin{figure}[!htb]
  \centering
    \includegraphics[scale=0.6]{../shared_images/business-logic/client-server-business-2.jpg}
   \caption{Полностью заполненный слой бизнес-логики}
    \label{fig:start}
\end{figure}

Тем не менее, идеальной моделью в данном случае является консолидированная модель, приведённая выше, когда вся бизнес-логика выделена из БД. Такая разработка имеет следующие преимущества:

\begin{itemize}
  \item Вся бизнес-логика находится в одном месте и может быть легко проверена, отлажена и изменена.
  \item Для реализации бизнес правил может быть использован нормальный язык разработки. В нашем случае таким языком стал Python, который больше подходит для данной задачи, чем SQL и хранимые процедуры.
  \item База данных становится слоем хранения и может заниматься эффективным получением и хранением данных без ограничений относящихся к слою бизнес-логики или представления.
\end{itemize}

Кроме того, основным преимуществом становится полное отделение от БД: вся работа с БД ведётся через адаптер, без привязки к конкретной технологии, будь то стэк Oracle или PostgreSQL. При необходимости могут быть составлены механизмы миграции, но бизнес-логика останется в неизменном виде. Как уже было сказано, это снимает лишнюю нагрузку с БД, что, в свою очередь, открывает широкие просторы для масштабирования:

\begin{figure}[!htb]
  \centering
    \includegraphics[scale=0.5]{../shared_images/business-logic/scaling.jpg}
   \caption{Поток данных и нагрузка}
    \label{fig:start}
\end{figure}

Перемещая вычисления в среднее звено, команда разработчиков отходит от границ слоя данных, что является несомненным плюсом: в то время как бизнес-звено имеет возможности масштабирования (можно закупить дополнительные сервера для обработки данных), попытка аналогичным образом увеличить количество серверов для хранения данных выльется в проблемы с дублированием, резервным копированием и -- возможно -- миграцией данных, что повлечёт замедление свежерасширенного слоя БД.

\addcontentsline{toc}{section}{Список использованных источников}
\begin{thebibliography}{}
\bibitem{} Электронный ресурс <<Википедия>>: \\https://ru.wikipedia.org/wiki/Бизнес-логика
\bibitem{} Chad Z. Hower ``Dude, where's my business logic?'': http://www.codeproject.com/Articles/10746/Dude-where-s-my-business-logic
\bibitem{} Руководство по проектированию реляционных баз данных: http://habrahabr.ru/post/193136/
\end{thebibliography}

\end{document}
