\documentclass[a4paper, 14pt]{extarticle}
\usepackage[left=2cm,right=2cm,top=2cm,bottom=2cm,bindingoffset=0cm]{geometry}
\usepackage[utf8]{inputenc}
\usepackage[english, russian]{babel}
\usepackage{amssymb, latexsym, amsmath}
\usepackage{indentfirst}
\usepackage{graphicx}
\usepackage{citehack}
\usepackage{tabularx}
\usepackage{listings}
\usepackage{pdfpages}

\lstloadlanguages{HTML,python}
\lstset{extendedchars=false,
  breaklines=true,
  breakatwhitespace=true,
  keepspaces = true,
  tabsize=2
}


\begin{document}
\begin{titlepage}

\newpage

\begin{center}
Московский Авиационный Институт \\*
(национальный исследовательский университет) \\*

\vspace{2em}

Факультет прикладной математики и физики \\*
Кафедра вычислительной математики и программирования

\vspace{10em}

\Large \textbf{Курсовой проект \\*
по дисциплине <<Информационные технологии в проектировании и производстве>>} \\*

\vspace{3em}

<<Разработка проектного офиса <<Цифромед>>
\end{center}

\vspace{8em}

\hspace{25em}\vbox{
  \hbox{\bfseries{Выполнил:}}
  \hbox{\hspace{1em} Данилычев И.\,А.}
}

\vspace{2em}

\hspace{25em}\vbox{
  \hbox{\bfseries{Руководитель:}}
  \hbox{\hspace{1em} Скородумов С.\,В.}
}

\vspace{\fill}

\begin{center}
Москва, 2015
\end{center}

\end{titlepage}

\newpage


\tableofcontents
\newpage


\section{Аннотация}
При компании <<Скородумов и Партнеры>> был сформирован отдел проектирования программного обеспечения, специализирующийся на создании интернет-решений для малого и среднего бизнеса, образовательных учреждений, госпредприятий, задачей которого стала разработка офиса управления проектами, позиционирующегося как SaaS-приложение (Software as a Service, или <<ПО как услуга>>).

Результатом работы над проектом стало появление на свет проектного офиса <<Цифромед>>, представленного в виде веб-приложения, реализованного на языке Python с применением многофункционального фреймворка Flask.

Одной из целей проекта также являлось обеспечение компании <<Скородумов и Партнеры>> решением для управления проектами, предназначенным для внутреннего пользования и при этом, несмотря на имеющиеся аналоги (teamtools.ru, <<Мегаплан>> и др.), не уступающим им по основным показателям производительности и функциональности.


В данный отчёт входят:

\begin{itemize}
\setlength{\itemsep}{-1mm}
\item Иерархическая модель работ (WBS, Work Breakdown Structure) и бизнес-модель проекта;
\item IDEF0-модель цикла разработки данного проекта;
\item Техническое задание;
\item Структура базы данных приложения;
\item Программная логика серверной части (backend);
\item Описание интерфейсной части (frontend);
\item Паспорт проекта и его данные.
\end{itemize}

\newpage


\section{Анализ проекта}
  \subsection{Цель проекта}
    \subsubsection{Достижимость цели}
    % Упомянуть о том, что <<мы не про>>
		
  \subsection{Продукт проекта}
  % Описать проектный офис. Взять универсальное описание откуда-л.
  
  \subsection{Work Breakdown Structure}
  % Описание, схема
  
  \subsection{Бизнес-модель проекта}
  % Описание, модель
  
  \subsection{IDEF0}
  % Описание, диаграмма
  
  \subsection{DFD}
  % Описание, диаграмма

\newpage


\section{Техническое задание}
% Краткая аннотация, а затем -- задание из репозитория (с небольшими переработками)

\newpage


\section{Структура приложения}
  \subsection{Схема}
  % Описание, <<как пришли>> и схема
  
  \subsection{Бизнес-логика}
  % Много из http://habrahabr.ru/post/65432/
  
\newpage


\section{Серверная часть} % исправить заголовок?

\newpage


\section{Клиентская часть} % исправить заголовок?

\newpage


\section{Выводы}
% Ещё немного про то, что мы, безусловно, не конкуренты кому-либо, но наш проект можно использовать и вообще!

\newpage


\section{Паспорт проекта}
% Сырые PDF, экспортированные из MS Word; я не хочу перевёрстывать их в TeX

\end{document}